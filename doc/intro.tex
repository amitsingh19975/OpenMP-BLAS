\chapter{Introduction}

Linear algebra is a vital tool in the toolbox for various applications, 
from solving a simple equation to the art of Deep Learning algorithm or 
Genomics. The impact can felt across modern-day inventions or day to day 
life. Many tried to optimize these routines by hand-coding them in the assembly 
or the compiler intrinsics to squeeze every bit of performance out of the CPU; 
some chip manufacturers provide library specific to their chip. A few high-quality 
libraries, such as \textbf{Intel's MKL}, \textbf{OpenBLAS}, \textbf{Flame's Blis}, 
\textbf{Eigen}, and more, each one has one common problem, 
they are architecture-specific. They need to be hand-tuned for each different architecture.

There three ways to implement the routines and each one has its shortcomings:
\begin{enumerate}
    \item Hand code them in assembly and try to optimize them for each architecture.
    \item Use the compiler intrinsics.
    \item Simply code them and let the compiler optimize.
\end{enumerate}

\section{Hand-tuned Assembly}

