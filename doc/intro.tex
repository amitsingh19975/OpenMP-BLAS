\chapter{Introduction}

Linear algebra is a vital tool in the toolbox for various applications, 
from solving a simple equation to the art of Deep Learning algorithm or 
Genomics. The impact can felt across modern-day inventions or day to day 
life. Many tried to optimize these routines by hand-coding them in the assembly 
or the compiler intrinsics to squeeze every bit of performance out of the CPU; 
some chip manufacturers provide library specific to their chip. A few high-quality 
libraries, such as \textbf{Intel's MKL}, \textbf{OpenBLAS}, \textbf{Flame's Blis}, 
\textbf{Eigen}, and more, each one has one common problem, 
they are architecture-specific. They need to be hand-tuned for each different architecture.

There three ways to implement the routines and each one has its shortcomings:
\begin{enumerate}
    \item Hand code them in assembly and try to optimize them for each architecture.
    \item Use the compiler intrinsics.
    \item Simply code them and let the compiler optimize.
\end{enumerate}

\section{Hand-tuned Assembly}

Hand-coded assembly gives high performance due to more control over registers, 
caches, and instructions used, but that comes with its issues, 
which we exchange for performance:

\begin{itemize}
    \item Need a deep understanding of the architecture
    \item Maintenance of the code
    \item It violates the DRY principle because we need to implement 
        the same algorithm for a different architecture
    \item Development time is high
    \item Debugging is hard
    \item Unreadable code
    \item Sometimes lead to micro-optimization or worst performance than the compiler
\end{itemize}

\textbf{Intel's MKL}, \textbf{OpenBLAS} and \textbf{Flame's Blis} comes under this category


\section{Compiler Intrinsics}

The compiler intrinsics is one layer above the assembly, 
and we lose control over which register to use. 
If an intrinsic has multiple representations, then we 
do not know which instruction will emit. 
There are the following issues with this approach:

\begin{itemize}
    \item Need the knowledge of intrinsic
    \item Maintenance of the code much better than assembly but not great
    \item DRY principle achieved with little abstraction
    \item Development time is better than assembly but not great
    \item Debugging is still hard
    \item Unreadable code if not careful
    \item May lead to micro-optimization or worst performance than the compiler
\end{itemize}

\textbf{Eigen} uses the compiler intrinsics, and they fixed and avoided some of the above problems with the right abstraction.

\section{Compiler Dependent Optimization}

The compiler has many tools for optimizing code: loop-unrolling, \\
auto-vectorization, inlining functions, etc. We will rely on code vectorization heavily, 
but auto-vectorization may or may not be applied if the compiler can 
infer enough information from the code. To avoid unreliable auto-vectorization, 
we will use \textbf{OpenMP} for explicit vectorization, an open standard and supported by powerful compilers. 
The main issues are:

\begin{itemize}
    \item Performance depends on the compiler
    \item No control over vector instructions or registers
    \item Auto-vectorization may fail
\end{itemize}

\textbf{Boost.uBLAS} depends on the auto-vectorization, 
which does not guarantee the code will vectorize.

\section{BLAS Routines}

There are four BLAS routines that we will implement using \textbf{OpenMP}, 
which uses explicit vectorization and parallelization using threads. 
Each routine has its chapter, and there we go much deeper with performance metrics.

\begin{enumerate}
    \item Vector-Vector Inner Product (\textbf{?dot})
    \item Vector-Vector Outer Product (\textbf{?ger})
    \item Matrix-Vector Product or Vector-Matrix Product (\textbf{?gemv})
    \item Matrix-Matrix Product (\textbf{?gemm})
\end{enumerate}

\section{Machine Model}

The machine model that we will follow is similar to the model defined in the 
\citep{BLIS}, which takes modern hardware into mind. Such as vector registers 
and a memory hierarchy with multiple levels of set-associative data caches. 
However, we will ignore vector registers because we let the \textbf{OpenMP} handle 
the registers, and we do not have any control over them. However, 
we will add multiple cores where each core has at least one cache level 
that not shared among the other cores.
The only parameter we need to put all our energy in is the cache hierarchy and 
how we can optimize the cache misses.

All the data caches are set-associative and we can characterize them based 
on the four parameter defined bellow:

\(C_{L_i}\): cache line of the \(i^{th}\) level

\(W_{L_i}\): associative degree of the \(i^{th}\) level

\(N_{L_i}\): Number of sets in the \(i^{th}\) level

\(S_{L_i}\): size of the \(i^{th}\) level in Bytes

\begin{equation}
    S_{L_i} = C_{L_i}W_{L_i}N_{L_i}
    \label{eqcache_size}
\end{equation}

Let the $S_{data}$ be the width of the type in Bytes.

We are assuming that the cache replacement policy for all cache levels is 
\textbf{LRU}, which also assumed in the \citep{BLIS} and the cache 
line is same for all the cache levels. For most of the case, 
we will try to avoid the associative so that we could derive 
a simple equation containing the cache size only from the equation \ref{eqcache_size}.

\section{Performance Metrics}

\subsection{FLOPS}

It represents the number of floating-point operations that a processor can perform per second. 
The higher the Flops are, the faster it achieves the floating-point specific operations, 
but we should not depend on the flops all the time because it might be deceiving. 
Moreover, it does not paint the whole picture.

\begin{equation}
    FLOPS = \frac{Number\ of\ Operation}{Time\ taken}
    \label{eqflops}
\end{equation}

\subsection{Speedup}

The speedup tells us how much performance we were able to get when compared to 
the existing implementation. If it is more significant than one, 
then reference implementation performs better than the existing implementation; 
otherwise, if it is less than one, reference implementation performs worse 
than the existing implementation.

\begin{equation}
    Speedup = \frac{Flops_{reference}}{Flops_{existing}}
    \label{eqspeedup}
\end{equation}

\subsection{Speed-down}

The speed down is the inverse of the speedup, and if it is below one, 
then we performing better than the existing implementation; 
otherwise, we are performing worse.

\begin{equation}
    Speeddown = \frac{Flops_{existing}}{Flops_{reference}}
    \label{eqspeeddown}
\end{equation}

\subsection{Peak Utilization}

This tells us how much CPU we are utilizing for floating-point operations when 
the CPU can compute X amount of floating-point operations.

\begin{equation}
    Peak\ Utilization = \frac{Flops}{Peak\ Performance} \times 100
    \label{eqpeak_util}
\end{equation}

\section{System Information}

\begin{tabular}{|l | l|}
    \hline
    Processor & 2.3 GHz 8-Core Intel Core i9 \\
    \hline
    Architecture & x86 \\
    \hline
    L1 Cache & 8-way, 32KiB \\
    \hline
    L2 Cache & 4-way, 256KiB \\
    \hline
    L3 Cache & 16-way, 16MiB \\
    \hline
    Cache Line & 64B \\
    \hline
    Single-Precision(FP32) & 32 FLOPs per cycle per core \\
    \hline
    Double-Precision(FP64) & 16 FLOPs per cycle per core\\
    \hline
\end{tabular}

Peak Utilization found using the formula defined on the \citep{wiki:FLOPS}

\begin{equation}
    Peak\ Utilization = Frequency \times Cores \times \frac{FLOPS}{Cycle}
\end{equation}

\[Peak\ Utilization_{FP32} = 588.8\ GFLOPS\]
\[Peak\ Utilization_{FP64} = 294.4\ GFLOPS\]

\section{Compiler Information}

\begin{tabular}{|l | l|}
    \hline
    Compiler & Apple clang version 12.0.0 (clang-1200.0.32.29) \\
    \hline
    Compiler Flags & -march=native -ffast-math -Xclang -fopenmp -O3 \\
    \hline
    C++ Standard & 20 \\
    \hline
\end{tabular}

\section{library Version}

\begin{tabular}{|l | l|}
    \hline
    Intel MKL & 2020.0.1 \\
    \hline
    Eigen & 3.3.9 \\
    \hline
    OpenBLAS & 0.3.13 \\
    \hline
    BLIS & 0.8.0 \\
    \hline
\end{tabular}